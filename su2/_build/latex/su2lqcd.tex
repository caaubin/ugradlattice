%% Generated by Sphinx.
\def\sphinxdocclass{report}
\documentclass[letterpaper,10pt,english]{sphinxmanual}
\ifdefined\pdfpxdimen
   \let\sphinxpxdimen\pdfpxdimen\else\newdimen\sphinxpxdimen
\fi \sphinxpxdimen=.75bp\relax

\PassOptionsToPackage{warn}{textcomp}
\usepackage[utf8]{inputenc}
\ifdefined\DeclareUnicodeCharacter
% support both utf8 and utf8x syntaxes
  \ifdefined\DeclareUnicodeCharacterAsOptional
    \def\sphinxDUC#1{\DeclareUnicodeCharacter{"#1}}
  \else
    \let\sphinxDUC\DeclareUnicodeCharacter
  \fi
  \sphinxDUC{00A0}{\nobreakspace}
  \sphinxDUC{2500}{\sphinxunichar{2500}}
  \sphinxDUC{2502}{\sphinxunichar{2502}}
  \sphinxDUC{2514}{\sphinxunichar{2514}}
  \sphinxDUC{251C}{\sphinxunichar{251C}}
  \sphinxDUC{2572}{\textbackslash}
\fi
\usepackage{cmap}
\usepackage[T1]{fontenc}
\usepackage{amsmath,amssymb,amstext}
\usepackage{babel}



\usepackage{times}
\expandafter\ifx\csname T@LGR\endcsname\relax
\else
% LGR was declared as font encoding
  \substitutefont{LGR}{\rmdefault}{cmr}
  \substitutefont{LGR}{\sfdefault}{cmss}
  \substitutefont{LGR}{\ttdefault}{cmtt}
\fi
\expandafter\ifx\csname T@X2\endcsname\relax
  \expandafter\ifx\csname T@T2A\endcsname\relax
  \else
  % T2A was declared as font encoding
    \substitutefont{T2A}{\rmdefault}{cmr}
    \substitutefont{T2A}{\sfdefault}{cmss}
    \substitutefont{T2A}{\ttdefault}{cmtt}
  \fi
\else
% X2 was declared as font encoding
  \substitutefont{X2}{\rmdefault}{cmr}
  \substitutefont{X2}{\sfdefault}{cmss}
  \substitutefont{X2}{\ttdefault}{cmtt}
\fi


\usepackage[Bjarne]{fncychap}
\usepackage{sphinx}

\fvset{fontsize=\small}
\usepackage{geometry}


% Include hyperref last.
\usepackage{hyperref}
% Fix anchor placement for figures with captions.
\usepackage{hypcap}% it must be loaded after hyperref.
% Set up styles of URL: it should be placed after hyperref.
\urlstyle{same}

\addto\captionsenglish{\renewcommand{\contentsname}{Contents:}}

\usepackage{sphinxmessages}
\setcounter{tocdepth}{1}



\title{SU(2) LQCD}
\date{Jun 25, 2020}
\release{}
\author{Sean Hannaford}
\newcommand{\sphinxlogo}{\vbox{}}
\renewcommand{\releasename}{}
\makeindex
\begin{document}

\pagestyle{empty}
\sphinxmaketitle
\pagestyle{plain}
\sphinxtableofcontents
\pagestyle{normal}
\phantomsection\label{\detokenize{index::doc}}

\index{module@\spxentry{module}!su2@\spxentry{su2}}\index{su2@\spxentry{su2}!module@\spxentry{module}}\index{calcPlaq() (in module su2)@\spxentry{calcPlaq()}\spxextra{in module su2}}

\begin{fulllineitems}
\phantomsection\label{\detokenize{index:su2.calcPlaq}}\pysiglinewithargsret{\sphinxcode{\sphinxupquote{su2.}}\sphinxbfcode{\sphinxupquote{calcPlaq}}}{\emph{\DUrole{n}{U}}, \emph{\DUrole{n}{V}}, \emph{\DUrole{n}{mups}}}{}
Calculates the average value of the plaquettes about all points
in the lattice
\begin{quote}\begin{description}
\item[{Parameters}] \leavevmode\begin{itemize}
\item {} 
\sphinxstyleliteralstrong{\sphinxupquote{U}} (\sphinxstyleliteralemphasis{\sphinxupquote{array\_like}}) \textendash{} Array containing the gaugefields for every point on the lattice

\item {} 
\sphinxstyleliteralstrong{\sphinxupquote{V}} (\sphinxstyleliteralemphasis{\sphinxupquote{int}}) \textendash{} The volume of the lattice, which is equivalent to the number of
points on the lattice

\item {} 
\sphinxstyleliteralstrong{\sphinxupquote{mups}} (\sphinxstyleliteralemphasis{\sphinxupquote{array\_like}}) \textendash{} The mups array. This array is used as shorthand for taking a
step forwards in the mu’th direction from the U0i’th point

\end{itemize}

\item[{Returns}] \leavevmode
The average value of the plaquettes about the whole lattice

\item[{Return type}] \leavevmode
numpy.float64

\end{description}\end{quote}

\end{fulllineitems}

\index{calcU\_i() (in module su2)@\spxentry{calcU\_i()}\spxextra{in module su2}}

\begin{fulllineitems}
\phantomsection\label{\detokenize{index:su2.calcU_i}}\pysiglinewithargsret{\sphinxcode{\sphinxupquote{su2.}}\sphinxbfcode{\sphinxupquote{calcU\_i}}}{\emph{\DUrole{n}{U}}, \emph{\DUrole{n}{V}}, \emph{\DUrole{n}{La}}, \emph{\DUrole{n}{mups}}}{}
Calculates the average values of the spacial links in the
lattice
\begin{quote}\begin{description}
\item[{Parameters}] \leavevmode\begin{itemize}
\item {} 
\sphinxstyleliteralstrong{\sphinxupquote{U}} (\sphinxstyleliteralemphasis{\sphinxupquote{array\_like}}) \textendash{} Array containing the gaugefields for every point on the lattice

\item {} 
\sphinxstyleliteralstrong{\sphinxupquote{V}} (\sphinxstyleliteralemphasis{\sphinxupquote{int}}) \textendash{} The volume of the lattice, which is equivalent to the number of
points on the lattice

\item {} 
\sphinxstyleliteralstrong{\sphinxupquote{mups}} (\sphinxstyleliteralemphasis{\sphinxupquote{array\_like}}) \textendash{} The mups array. This array is used as shorthand for taking a
step forwards in the mu’th direction from the U0i’th point

\end{itemize}

\item[{Returns}] \leavevmode
The average value of the spacial links in the lattice

\item[{Return type}] \leavevmode
numpy.float64

\end{description}\end{quote}

\end{fulllineitems}

\index{calcU\_t() (in module su2)@\spxentry{calcU\_t()}\spxextra{in module su2}}

\begin{fulllineitems}
\phantomsection\label{\detokenize{index:su2.calcU_t}}\pysiglinewithargsret{\sphinxcode{\sphinxupquote{su2.}}\sphinxbfcode{\sphinxupquote{calcU\_t}}}{\emph{\DUrole{n}{U}}, \emph{\DUrole{n}{V}}, \emph{\DUrole{n}{mups}}}{}
Calculates the average values of the time links in the lattice
\begin{quote}\begin{description}
\item[{Parameters}] \leavevmode\begin{itemize}
\item {} 
\sphinxstyleliteralstrong{\sphinxupquote{U}} (\sphinxstyleliteralemphasis{\sphinxupquote{array\_like}}) \textendash{} Array containing the gaugefields for every point on the lattice

\item {} 
\sphinxstyleliteralstrong{\sphinxupquote{V}} (\sphinxstyleliteralemphasis{\sphinxupquote{int}}) \textendash{} The volume of the lattice, which is equivalent to the number of
points on the lattice

\item {} 
\sphinxstyleliteralstrong{\sphinxupquote{mups}} (\sphinxstyleliteralemphasis{\sphinxupquote{array\_like}}) \textendash{} The mups array. This array is used as shorthand for taking a
step forwards in the mu’th direction from the U0i’th point

\end{itemize}

\item[{Returns}] \leavevmode
The average value of the time links in the lattice

\item[{Return type}] \leavevmode
numpy.float64

\end{description}\end{quote}

\end{fulllineitems}

\index{compare() (in module su2)@\spxentry{compare()}\spxextra{in module su2}}

\begin{fulllineitems}
\phantomsection\label{\detokenize{index:su2.compare}}\pysiglinewithargsret{\sphinxcode{\sphinxupquote{su2.}}\sphinxbfcode{\sphinxupquote{compare}}}{\emph{\DUrole{n}{mat1}}, \emph{\DUrole{n}{mat2}}}{}
Prints any dissimilar elements between two matrices
\begin{quote}\begin{description}
\item[{Parameters}] \leavevmode\begin{itemize}
\item {} 
\sphinxstyleliteralstrong{\sphinxupquote{mat1}} (\sphinxstyleliteralemphasis{\sphinxupquote{array\_like}}) \textendash{} A matrix

\item {} 
\sphinxstyleliteralstrong{\sphinxupquote{mat2}} (\sphinxstyleliteralemphasis{\sphinxupquote{array\_like}}) \textendash{} A matrix

\end{itemize}

\item[{Returns}] \leavevmode
Compares corresponding elements between two matrices and prints
any values which are not exactly equal.

\item[{Return type}] \leavevmode
void

\end{description}\end{quote}

\end{fulllineitems}

\index{corr() (in module su2)@\spxentry{corr()}\spxextra{in module su2}}

\begin{fulllineitems}
\phantomsection\label{\detokenize{index:su2.corr}}\pysiglinewithargsret{\sphinxcode{\sphinxupquote{su2.}}\sphinxbfcode{\sphinxupquote{corr}}}{\emph{\DUrole{n}{x}}}{}
Writes out the correlator to a file
\begin{quote}\begin{description}
\item[{Parameters}] \leavevmode
\sphinxstyleliteralstrong{\sphinxupquote{x}} (\sphinxstyleliteralemphasis{\sphinxupquote{array\_like}}) \textendash{} The inverse

\end{description}\end{quote}

\end{fulllineitems}

\index{cstart() (in module su2)@\spxentry{cstart()}\spxextra{in module su2}}

\begin{fulllineitems}
\phantomsection\label{\detokenize{index:su2.cstart}}\pysiglinewithargsret{\sphinxcode{\sphinxupquote{su2.}}\sphinxbfcode{\sphinxupquote{cstart}}}{}{}
Returns 2x2 identity matrix
\begin{quote}\begin{description}
\item[{Returns}] \leavevmode
2x2 Identity matrix written in the convention of a real\sphinxhyphen{}valued
SU(2) matrix

\item[{Return type}] \leavevmode
numpy.ndarray

\end{description}\end{quote}

\end{fulllineitems}

\index{dag() (in module su2)@\spxentry{dag()}\spxextra{in module su2}}

\begin{fulllineitems}
\phantomsection\label{\detokenize{index:su2.dag}}\pysiglinewithargsret{\sphinxcode{\sphinxupquote{su2.}}\sphinxbfcode{\sphinxupquote{dag}}}{\emph{\DUrole{n}{U}}}{}
Gives the hermitian conjugate of a matrix writen in the
real\sphinxhyphen{}valued representation of an SU(2) matrix
\begin{quote}\begin{description}
\item[{Parameters}] \leavevmode
\sphinxstyleliteralstrong{\sphinxupquote{u}} (\sphinxstyleliteralemphasis{\sphinxupquote{array\_like}}) \textendash{} Real\sphinxhyphen{}valued matrix representaion of an SU(2) matrix

\item[{Returns}] \leavevmode
Hermitian conjugate of the input written as a real\sphinxhyphen{}valued
representation of an SU(2) matrix

\item[{Return type}] \leavevmode
numpy.ndarray

\end{description}\end{quote}

\end{fulllineitems}

\index{dagger() (in module su2)@\spxentry{dagger()}\spxextra{in module su2}}

\begin{fulllineitems}
\phantomsection\label{\detokenize{index:su2.dagger}}\pysiglinewithargsret{\sphinxcode{\sphinxupquote{su2.}}\sphinxbfcode{\sphinxupquote{dagger}}}{\emph{\DUrole{n}{u}}}{}
Gives the hermitian conjugate of a matrix
\begin{quote}\begin{description}
\item[{Parameters}] \leavevmode
\sphinxstyleliteralstrong{\sphinxupquote{u}} (\sphinxstyleliteralemphasis{\sphinxupquote{array\_like}}) \textendash{} Matrix representing a gauge field

\item[{Returns}] \leavevmode
Hermitian conjugate of the input

\item[{Return type}] \leavevmode
numpy.ndarray

\end{description}\end{quote}

\end{fulllineitems}

\index{det() (in module su2)@\spxentry{det()}\spxextra{in module su2}}

\begin{fulllineitems}
\phantomsection\label{\detokenize{index:su2.det}}\pysiglinewithargsret{\sphinxcode{\sphinxupquote{su2.}}\sphinxbfcode{\sphinxupquote{det}}}{\emph{\DUrole{n}{UU}}}{}
Returns the determinant of the matrix
\begin{quote}\begin{description}
\item[{Parameters}] \leavevmode
\sphinxstyleliteralstrong{\sphinxupquote{UU}} (\sphinxstyleliteralemphasis{\sphinxupquote{array\_like}}) \textendash{} SU(2) matrix writen in real\sphinxhyphen{}valued form

\item[{Returns}] \leavevmode
Determinant of the input matrix

\item[{Return type}] \leavevmode
numpy.float64

\end{description}\end{quote}

\end{fulllineitems}

\index{dim() (in module su2)@\spxentry{dim()}\spxextra{in module su2}}

\begin{fulllineitems}
\phantomsection\label{\detokenize{index:su2.dim}}\pysiglinewithargsret{\sphinxcode{\sphinxupquote{su2.}}\sphinxbfcode{\sphinxupquote{dim}}}{\emph{\DUrole{n}{La}}}{}
Returns the dimensions of the array in a dictionary
\begin{quote}\begin{description}
\item[{Parameters}] \leavevmode
\sphinxstyleliteralstrong{\sphinxupquote{La}} (\sphinxstyleliteralemphasis{\sphinxupquote{array\_like}}) \textendash{} Array where each element describes the length of one
dimension of the lattice ({[}x,y,z,t{]})

\item[{Returns}] \leavevmode
Dictionary containing the lattice dimensions

\item[{Return type}] \leavevmode
dict

\end{description}\end{quote}

\end{fulllineitems}

\index{getElement() (in module su2)@\spxentry{getElement()}\spxextra{in module su2}}

\begin{fulllineitems}
\phantomsection\label{\detokenize{index:su2.getElement}}\pysiglinewithargsret{\sphinxcode{\sphinxupquote{su2.}}\sphinxbfcode{\sphinxupquote{getElement}}}{\emph{\DUrole{n}{D}}, \emph{\DUrole{n}{La}}, \emph{\DUrole{n}{point\_1}}, \emph{\DUrole{n}{point\_2}}, \emph{\DUrole{n}{a}}, \emph{\DUrole{n}{alpha}}, \emph{\DUrole{n}{b}}, \emph{\DUrole{n}{beta}}}{}
input the matrix, the dimensions of the latice, the initial and final point of the particle,
and the other spin and field stuff, and get the associated element of the matrix
point\_1, point\_2 range from 0 to {[}max index{]} \sphinxhyphen{} 1
a, b range from 0 to 1
alpha, beta range from 0 to 3

\end{fulllineitems}

\index{getIndex() (in module su2)@\spxentry{getIndex()}\spxextra{in module su2}}

\begin{fulllineitems}
\phantomsection\label{\detokenize{index:su2.getIndex}}\pysiglinewithargsret{\sphinxcode{\sphinxupquote{su2.}}\sphinxbfcode{\sphinxupquote{getIndex}}}{\emph{\DUrole{n}{La}}, \emph{\DUrole{n}{point\_1}}}{}
Returns the index of the full Dirac matrix which corresponds to
the element associated with a particular point on the lattice
\begin{quote}\begin{description}
\item[{Parameters}] \leavevmode\begin{itemize}
\item {} 
\sphinxstyleliteralstrong{\sphinxupquote{La}} (\sphinxstyleliteralemphasis{\sphinxupquote{array\_like}}) \textendash{} Array where each element describes the length of one
dimension of the lattice ({[}x,y,z,t{]})

\item {} 
\sphinxstyleliteralstrong{\sphinxupquote{point}} (\sphinxstyleliteralemphasis{\sphinxupquote{array\_like}}) \textendash{} Array containing the spacetime coordinates of a position on
the lattice written as {[}x,y,z,t{]}

\end{itemize}

\item[{Returns}] \leavevmode
This function operates very similarly to p2i

\item[{Return type}] \leavevmode
int

\end{description}\end{quote}

\end{fulllineitems}

\index{getMups() (in module su2)@\spxentry{getMups()}\spxextra{in module su2}}

\begin{fulllineitems}
\phantomsection\label{\detokenize{index:su2.getMups}}\pysiglinewithargsret{\sphinxcode{\sphinxupquote{su2.}}\sphinxbfcode{\sphinxupquote{getMups}}}{\emph{\DUrole{n}{V}}, \emph{\DUrole{n}{numdim}}, \emph{\DUrole{n}{La}}}{}
Returns the mups array
\begin{quote}\begin{description}
\item[{Parameters}] \leavevmode\begin{itemize}
\item {} 
\sphinxstyleliteralstrong{\sphinxupquote{V}} (\sphinxstyleliteralemphasis{\sphinxupquote{int}}) \textendash{} The volume of the lattice, which is equivalent to the number of
points on the lattice

\item {} 
\sphinxstyleliteralstrong{\sphinxupquote{numdim}} (\sphinxstyleliteralemphasis{\sphinxupquote{int}}) \textendash{} Number of dimensions of the lattice

\item {} 
\sphinxstyleliteralstrong{\sphinxupquote{La}} (\sphinxstyleliteralemphasis{\sphinxupquote{array\_like}}) \textendash{} Array where each element describes the length of one
dimension of the lattice ({[}x,y,z,t{]})

\end{itemize}

\item[{Returns}] \leavevmode
The output is a V x numdim matrix array which can be used as
shorthand when calling elements from the gaugefield array.
Specifically, this array can be used in functions which
involve stepping up or down between points on the lattice.
In the output matrix array, the ith array corresponds to the
ith point on the lattice, and the elements of that array are
the indexes of the adjacent points. For example, the {[}i,mu{]}
element in the output array is the index of point on the
lattice corresponding to the point one step in the mu’th
direction.

\item[{Return type}] \leavevmode
numpy.ndarray

\end{description}\end{quote}

\end{fulllineitems}

\index{getTime() (in module su2)@\spxentry{getTime()}\spxextra{in module su2}}

\begin{fulllineitems}
\phantomsection\label{\detokenize{index:su2.getTime}}\pysiglinewithargsret{\sphinxcode{\sphinxupquote{su2.}}\sphinxbfcode{\sphinxupquote{getTime}}}{\emph{\DUrole{n}{a}}, \emph{\DUrole{n}{b}}}{}
Displays the elalpsed time in hr:min:sec format
\begin{quote}\begin{description}
\item[{Parameters}] \leavevmode\begin{itemize}
\item {} 
\sphinxstyleliteralstrong{\sphinxupquote{a}} (\sphinxstyleliteralemphasis{\sphinxupquote{float}}) \textendash{} The current time given by the time() function in the standard
time module

\item {} 
\sphinxstyleliteralstrong{\sphinxupquote{b}} (\sphinxstyleliteralemphasis{\sphinxupquote{float}}) \textendash{} The current time given by the time() function in the standard
time module

\end{itemize}

\item[{Returns}] \leavevmode
Calculates the time elapsed between when a and b were
initialized and print the elapsed time in hr:min:sec format.
Can be used to find the time elapsed by a section of code if a
is initialized before said section and b is intialized
immediately after

\item[{Return type}] \leavevmode
string

\end{description}\end{quote}

\end{fulllineitems}

\index{getstaple() (in module su2)@\spxentry{getstaple()}\spxextra{in module su2}}

\begin{fulllineitems}
\phantomsection\label{\detokenize{index:su2.getstaple}}\pysiglinewithargsret{\sphinxcode{\sphinxupquote{su2.}}\sphinxbfcode{\sphinxupquote{getstaple}}}{\emph{\DUrole{n}{U}}, \emph{\DUrole{n}{U0i}}, \emph{\DUrole{n}{mups}}, \emph{\DUrole{n}{mdns}}, \emph{\DUrole{n}{mu}}}{}
Returns the value of the staple
\begin{quote}\begin{description}
\item[{Parameters}] \leavevmode\begin{itemize}
\item {} 
\sphinxstyleliteralstrong{\sphinxupquote{U}} (\sphinxstyleliteralemphasis{\sphinxupquote{array\_like}}) \textendash{} Array containing the gaugefields for every point on the lattice

\item {} 
\sphinxstyleliteralstrong{\sphinxupquote{U0i}} (\sphinxstyleliteralemphasis{\sphinxupquote{int}}) \textendash{} Lattice point index of the starting point on the lattice for
the calculation.

\item {} 
\sphinxstyleliteralstrong{\sphinxupquote{mups}} (\sphinxstyleliteralemphasis{\sphinxupquote{array\_like}}) \textendash{} The mups array. This array is used as shorthand for taking a
step forwards in the mu’th direction from the U0i’th point

\item {} 
\sphinxstyleliteralstrong{\sphinxupquote{mdns}} (\sphinxstyleliteralemphasis{\sphinxupquote{array\_like}}) \textendash{} The mdns array. This array is used as shorthand for taking a
step backwards in the mu’th direction from the U0i’th point

\item {} 
\sphinxstyleliteralstrong{\sphinxupquote{mu}} (\sphinxstyleliteralemphasis{\sphinxupquote{int}}) \textendash{} Index corresponding to one of the directions on the lattice:
0:x, 1:y, 2:z, 3:t

\end{itemize}

\item[{Returns}] \leavevmode
Returns the staple starting at the U0i’th point

\item[{Return type}] \leavevmode
numpy.ndarray

\end{description}\end{quote}

\end{fulllineitems}

\index{hstart() (in module su2)@\spxentry{hstart()}\spxextra{in module su2}}

\begin{fulllineitems}
\phantomsection\label{\detokenize{index:su2.hstart}}\pysiglinewithargsret{\sphinxcode{\sphinxupquote{su2.}}\sphinxbfcode{\sphinxupquote{hstart}}}{}{}
Returns a random complex 2x2 matrix written in real\sphinxhyphen{}valued form
\begin{quote}\begin{description}
\item[{Returns}] \leavevmode
2x2 matrix written in real\sphinxhyphen{}valued form. Elements are assigned
to random values between \sphinxhyphen{}1 and 1

\item[{Return type}] \leavevmode
numpy.ndarray

\end{description}\end{quote}

\end{fulllineitems}

\index{i2p() (in module su2)@\spxentry{i2p()}\spxextra{in module su2}}

\begin{fulllineitems}
\phantomsection\label{\detokenize{index:su2.i2p}}\pysiglinewithargsret{\sphinxcode{\sphinxupquote{su2.}}\sphinxbfcode{\sphinxupquote{i2p}}}{\emph{\DUrole{n}{ind}}, \emph{\DUrole{n}{La}}}{}
Takes the index of a point on the lattice and returns the
spacetime position of that point
\begin{quote}\begin{description}
\item[{Parameters}] \leavevmode\begin{itemize}
\item {} 
\sphinxstyleliteralstrong{\sphinxupquote{ind}} (\sphinxstyleliteralemphasis{\sphinxupquote{int}}) \textendash{} As the elements of the Dirac matrix are themselves 8x8
matrices, i is the index of the first row/colum of the elements
in the Dirac matrix which pertain to a particular spacetime
position

\item {} 
\sphinxstyleliteralstrong{\sphinxupquote{La}} (\sphinxstyleliteralemphasis{\sphinxupquote{array\_like}}) \textendash{} Array where each element describes the length of one dimension
of the lattice ({[}x,y,z,t{]})

\end{itemize}

\item[{Returns}] \leavevmode
The position on the lattice which corresponds to the input
index. The position is written as {[}x,y,z,t{]}

\item[{Return type}] \leavevmode
numpy.ndarray

\end{description}\end{quote}

\end{fulllineitems}

\index{initD() (in module su2)@\spxentry{initD()}\spxextra{in module su2}}

\begin{fulllineitems}
\phantomsection\label{\detokenize{index:su2.initD}}\pysiglinewithargsret{\sphinxcode{\sphinxupquote{su2.}}\sphinxbfcode{\sphinxupquote{initD}}}{\emph{\DUrole{n}{row}}, \emph{\DUrole{n}{col}}, \emph{\DUrole{n}{dat}}, \emph{\DUrole{n}{mu}}, \emph{\DUrole{n}{U}}, \emph{\DUrole{n}{r}}, \emph{\DUrole{n}{m}}, \emph{\DUrole{n}{n}}, \emph{\DUrole{n}{pbc}}}{}
Generates the data needed to make an 8x8 sparse submatrix
containing the kinetic terms (including time) of the dirac matrix.
\begin{quote}\begin{description}
\item[{Parameters}] \leavevmode\begin{itemize}
\item {} 
\sphinxstyleliteralstrong{\sphinxupquote{row}} (\sphinxstyleliteralemphasis{\sphinxupquote{list}}) \textendash{} list containing row indices for the non\sphinxhyphen{}zero elements of the
dirac matrix. Is used to generate the dirac matrix as a sparse
matrix.

\item {} 
\sphinxstyleliteralstrong{\sphinxupquote{col}} (\sphinxstyleliteralemphasis{\sphinxupquote{list}}) \textendash{} list containing column indices for the non\sphinxhyphen{}zero elements of the
dirac matrix. Is used to generate the dirac matrix as a sparse
matrix.

\item {} 
\sphinxstyleliteralstrong{\sphinxupquote{dat}} (\sphinxstyleliteralemphasis{\sphinxupquote{list}}) \textendash{} list containing the values of the non\sphinxhyphen{}zero elements of the
dirac matrix. Is used to generate the dirac matrix as a sparse
matrix.

\item {} 
\sphinxstyleliteralstrong{\sphinxupquote{mu}} (\sphinxstyleliteralemphasis{\sphinxupquote{int}}) \textendash{} Index corresponding to one of the directions on the lattice:
0:x, 1:y, 2:z, 3:t

\item {} 
\sphinxstyleliteralstrong{\sphinxupquote{U}} (\sphinxstyleliteralemphasis{\sphinxupquote{array\_like}}) \textendash{} Array containing the gaugefields for every point on the lattice

\item {} 
\sphinxstyleliteralstrong{\sphinxupquote{r}} (\sphinxstyleliteralemphasis{\sphinxupquote{double}}) \textendash{} the value of the wilson term

\item {} 
\sphinxstyleliteralstrong{\sphinxupquote{m}} (\sphinxstyleliteralemphasis{\sphinxupquote{int}}) \textendash{} row index of the Dirac matrix which, when used in conjunction
with n, will locate the first diagonal element of the submatrix

\item {} 
\sphinxstyleliteralstrong{\sphinxupquote{n}} (\sphinxstyleliteralemphasis{\sphinxupquote{int}}) \textendash{} column index of the Dirac matrix which, when used in
conjunction with m, will locate the first diagonal element of
the submatrix

\item {} 
\sphinxstyleliteralstrong{\sphinxupquote{pbc}} (\sphinxstyleliteralemphasis{\sphinxupquote{bool}}) \textendash{} boolean variable indicating whethere the particle has triggered
periodic boundary conditions. Is true when the particle hit a
boundary and had to be moved to the opposite end of the
lattice, and false otherwise.

\end{itemize}

\item[{Returns}] \leavevmode
Appends the row, col, and dat lists with the data needed to
construct an 8x8 submatrix which will be part of the larger
dirac matrix. The submatrix is the kronecker product of the
gamma matrix corresponding to the mu direction and the
gaugefield connecting the lattice points given by
su2.i2p(m//8,La) and su2.i2p(n//8,La). Returns error codes if
an error occurs or 0 otherwise, and the number of elements
initialized.

\item[{Return type}] \leavevmode
int, int

\end{description}\end{quote}

\end{fulllineitems}

\index{initDeo() (in module su2)@\spxentry{initDeo()}\spxextra{in module su2}}

\begin{fulllineitems}
\phantomsection\label{\detokenize{index:su2.initDeo}}\pysiglinewithargsret{\sphinxcode{\sphinxupquote{su2.}}\sphinxbfcode{\sphinxupquote{initDeo}}}{\emph{\DUrole{n}{row}}, \emph{\DUrole{n}{col}}, \emph{\DUrole{n}{dat}}, \emph{\DUrole{n}{mu}}, \emph{\DUrole{n}{U}}, \emph{\DUrole{n}{r}}, \emph{\DUrole{n}{m}}, \emph{\DUrole{n}{n}}, \emph{\DUrole{n}{pbc}}}{}
Generates the data needed to make an 8x8 sparse, even/odd
precoditioned submatrix containing the kinetic terms (including
time) of the dirac matrix.
\begin{quote}\begin{description}
\item[{Parameters}] \leavevmode\begin{itemize}
\item {} 
\sphinxstyleliteralstrong{\sphinxupquote{row}} (\sphinxstyleliteralemphasis{\sphinxupquote{list}}) \textendash{} list containing row indices for the non\sphinxhyphen{}zero elements of the
dirac matrix. Is used to generate the dirac matrix as a sparse
matrix.

\item {} 
\sphinxstyleliteralstrong{\sphinxupquote{col}} (\sphinxstyleliteralemphasis{\sphinxupquote{list}}) \textendash{} list containing column indices for the non\sphinxhyphen{}zero elements of the
dirac matrix. Is used to generate the dirac matrix as a sparse
matrix.

\item {} 
\sphinxstyleliteralstrong{\sphinxupquote{dat}} (\sphinxstyleliteralemphasis{\sphinxupquote{list}}) \textendash{} list containing the values of the non\sphinxhyphen{}zero elements of the
dirac matrix. Is used to generate the dirac matrix as a sparse
matrix.

\item {} 
\sphinxstyleliteralstrong{\sphinxupquote{mu}} (\sphinxstyleliteralemphasis{\sphinxupquote{int}}) \textendash{} Index corresponding to one of the directions on the lattice:
0:x, 1:y, 2:z, 3:t

\item {} 
\sphinxstyleliteralstrong{\sphinxupquote{U}} (\sphinxstyleliteralemphasis{\sphinxupquote{array\_like}}) \textendash{} Array containing the gaugefields for every point on the lattice

\item {} 
\sphinxstyleliteralstrong{\sphinxupquote{r}} (\sphinxstyleliteralemphasis{\sphinxupquote{double}}) \textendash{} the value of the wilson term

\item {} 
\sphinxstyleliteralstrong{\sphinxupquote{m}} (\sphinxstyleliteralemphasis{\sphinxupquote{int}}) \textendash{} row index of the Dirac matrix which, when used in conjunction
with n, will locate the first diagonal element of the submatrix

\item {} 
\sphinxstyleliteralstrong{\sphinxupquote{n}} (\sphinxstyleliteralemphasis{\sphinxupquote{int}}) \textendash{} column index of the Dirac matrix which, when used in
conjunction with m, will locate the first diagonal element of
the submatrix

\item {} 
\sphinxstyleliteralstrong{\sphinxupquote{pbc}} (\sphinxstyleliteralemphasis{\sphinxupquote{bool}}) \textendash{} boolean variable indicating whethere the particle has triggered
periodic boundary conditions. Is true when the particle hit a
boundary and had to be moved to the opposite end of the
lattice, and false otherwise.

\end{itemize}

\item[{Returns}] \leavevmode
This function takes advantage of even/odd preconditioning.
Since the Dirac matrix can be broken into mass terms and terms
which either connect “even” points with “odd” points or vice
versa, the Dirac matrix can be formulated as three smaller
matrices which are half the order of the full matrix. So, this
fucntion appends the row, col, and dat lists with the data
needed to construct an 8x8 submatrix which will be part of the
larger even\sphinxhyphen{}odd or odd\sphinxhyphen{}even precoditioned matrix. The submatrix
is the kronecker product of the gamma matrix corresponding to
the mu direction and the gaugefield connecting the lattice
points given by su2.i2p(m//8,La) and su2.i2p(n//8,La). Returns
error codes if an error occurs or 0 otherwise, and the number
of elements initialized.

\item[{Return type}] \leavevmode
int, int

\end{description}\end{quote}

\end{fulllineitems}

\index{link() (in module su2)@\spxentry{link()}\spxextra{in module su2}}

\begin{fulllineitems}
\phantomsection\label{\detokenize{index:su2.link}}\pysiglinewithargsret{\sphinxcode{\sphinxupquote{su2.}}\sphinxbfcode{\sphinxupquote{link}}}{\emph{\DUrole{n}{U}}, \emph{\DUrole{n}{U0i}}, \emph{\DUrole{n}{mups}}, \emph{\DUrole{n}{mu}}}{}
Returns the trace of the link between two points
\begin{quote}\begin{description}
\item[{Parameters}] \leavevmode\begin{itemize}
\item {} 
\sphinxstyleliteralstrong{\sphinxupquote{U}} (\sphinxstyleliteralemphasis{\sphinxupquote{array\_like}}) \textendash{} Array containing the gaugefields for every point on the lattice

\item {} 
\sphinxstyleliteralstrong{\sphinxupquote{U0i}} (\sphinxstyleliteralemphasis{\sphinxupquote{int}}) \textendash{} Lattice point index of the starting point on the lattice for
the calculation.

\item {} 
\sphinxstyleliteralstrong{\sphinxupquote{mups}} (\sphinxstyleliteralemphasis{\sphinxupquote{array\_like}}) \textendash{} The mups array. This array is used as shorthand for taking a
step forwards in the mu’th direction from the U0i’th point

\item {} 
\sphinxstyleliteralstrong{\sphinxupquote{mu}} (\sphinxstyleliteralemphasis{\sphinxupquote{int}}) \textendash{} Index corresponding to one of the directions on the lattice:
0:x, 1:y, 2:z, 3:t

\end{itemize}

\item[{Returns}] \leavevmode
The value of the link between the point at U0i and the point
one step in the mu’th direction

\item[{Return type}] \leavevmode
numpy.float64

\end{description}\end{quote}

\end{fulllineitems}

\index{masseo() (in module su2)@\spxentry{masseo()}\spxextra{in module su2}}

\begin{fulllineitems}
\phantomsection\label{\detokenize{index:su2.masseo}}\pysiglinewithargsret{\sphinxcode{\sphinxupquote{su2.}}\sphinxbfcode{\sphinxupquote{masseo}}}{\emph{\DUrole{n}{row}}, \emph{\DUrole{n}{dat}}, \emph{\DUrole{n}{i}}, \emph{\DUrole{n}{j}}, \emph{\DUrole{n}{m}}, \emph{\DUrole{n}{r}}}{}
Generates the data needed to make an 8x8 sparse submatrix
containing the mass terms of the dirac matrix
\begin{quote}\begin{description}
\item[{Parameters}] \leavevmode\begin{itemize}
\item {} 
\sphinxstyleliteralstrong{\sphinxupquote{row}} (\sphinxstyleliteralemphasis{\sphinxupquote{list}}) \textendash{} list containing row indices for the non\sphinxhyphen{}zero elements of the
dirac matrix. Is used to generate the dirac matrix as a sparse
matrix.

\item {} 
\sphinxstyleliteralstrong{\sphinxupquote{dat}} (\sphinxstyleliteralemphasis{\sphinxupquote{list}}) \textendash{} list containing the values of the non\sphinxhyphen{}zero elements of the
dirac matrix. Is used to generate the dirac matrix as a sparse
matrix.

\item {} 
\sphinxstyleliteralstrong{\sphinxupquote{i}} (\sphinxstyleliteralemphasis{\sphinxupquote{int}}) \textendash{} row index of the Dirac matrix which, when used in conjunction
with j, will locate the first diagonal element of the submatrix

\item {} 
\sphinxstyleliteralstrong{\sphinxupquote{j}} (\sphinxstyleliteralemphasis{\sphinxupquote{int}}) \textendash{} column index of the Dirac matrix which, when used in
conjunction with i, will locate the first diagonal element of
the submatrix

\item {} 
\sphinxstyleliteralstrong{\sphinxupquote{m}} (\sphinxstyleliteralemphasis{\sphinxupquote{double}}) \textendash{} the mass of the particle

\item {} 
\sphinxstyleliteralstrong{\sphinxupquote{r}} (\sphinxstyleliteralemphasis{\sphinxupquote{double}}) \textendash{} the value of the wilson term

\end{itemize}

\item[{Returns}] \leavevmode
Appends the row and dat lists with the data needed to construct
an 8x8 submatrix which will be part of the larger dirac matrix.
All mass terms should be along the diagonal of each submatrix
as well as the larger dirac matrix.

\item[{Return type}] \leavevmode
void

\end{description}\end{quote}

\end{fulllineitems}

\index{mdowni() (in module su2)@\spxentry{mdowni()}\spxextra{in module su2}}

\begin{fulllineitems}
\phantomsection\label{\detokenize{index:su2.mdowni}}\pysiglinewithargsret{\sphinxcode{\sphinxupquote{su2.}}\sphinxbfcode{\sphinxupquote{mdowni}}}{\emph{\DUrole{n}{ind}}, \emph{\DUrole{n}{mu}}, \emph{\DUrole{n}{La}}}{}
Decrement a position in the mu’th direction, looping if needed
\begin{quote}\begin{description}
\item[{Parameters}] \leavevmode\begin{itemize}
\item {} 
\sphinxstyleliteralstrong{\sphinxupquote{ind}} (\sphinxstyleliteralemphasis{\sphinxupquote{int}}) \textendash{} As the elements of the Dirac matrix are themselves 8x8
matrices, i is the index of the first row/colum of the elements
in the Dirac matrix which pertain to a particular spacetime
position

\item {} 
\sphinxstyleliteralstrong{\sphinxupquote{mu}} (\sphinxstyleliteralemphasis{\sphinxupquote{int}}) \textendash{} Index corresponding to one of the directions on the lattice:
0:x, 1:y, 2:z, 3:t

\item {} 
\sphinxstyleliteralstrong{\sphinxupquote{La}} (\sphinxstyleliteralemphasis{\sphinxupquote{array\_like}}) \textendash{} Array where each element describes the length of one dimension
of the lattice ({[}x,y,z,t{]})

\end{itemize}

\item[{Returns}] \leavevmode
The function deccrements a step in the mu’th direction, if the
boundary is met it then loops around to the other side of the
lattice. The return value is the index of the new point on the
lattice.

\item[{Return type}] \leavevmode
numpy.int64

\end{description}\end{quote}

\end{fulllineitems}

\index{mult() (in module su2)@\spxentry{mult()}\spxextra{in module su2}}

\begin{fulllineitems}
\phantomsection\label{\detokenize{index:su2.mult}}\pysiglinewithargsret{\sphinxcode{\sphinxupquote{su2.}}\sphinxbfcode{\sphinxupquote{mult}}}{\emph{\DUrole{n}{U1}}, \emph{\DUrole{n}{U2}}}{}
Multiplies two SU(2) matrices written in the real\sphinxhyphen{}valued
representation
\begin{quote}\begin{description}
\item[{Parameters}] \leavevmode\begin{itemize}
\item {} 
\sphinxstyleliteralstrong{\sphinxupquote{U1}} (\sphinxstyleliteralemphasis{\sphinxupquote{array\_like}}) \textendash{} Real\sphinxhyphen{}valued matrix representaion of an SU(2) gauge field

\item {} 
\sphinxstyleliteralstrong{\sphinxupquote{U2}} (\sphinxstyleliteralemphasis{\sphinxupquote{array\_like}}) \textendash{} Real\sphinxhyphen{}valued matrix representaion of an SU(2) gauge field

\end{itemize}

\item[{Returns}] \leavevmode
The product of the two input arrays written as a real\sphinxhyphen{}valued
matrix representaion of an SU(2) matrix

\item[{Return type}] \leavevmode
numpy.ndarray

\end{description}\end{quote}

\end{fulllineitems}

\index{mupi() (in module su2)@\spxentry{mupi()}\spxextra{in module su2}}

\begin{fulllineitems}
\phantomsection\label{\detokenize{index:su2.mupi}}\pysiglinewithargsret{\sphinxcode{\sphinxupquote{su2.}}\sphinxbfcode{\sphinxupquote{mupi}}}{\emph{\DUrole{n}{ind}}, \emph{\DUrole{n}{mu}}, \emph{\DUrole{n}{La}}}{}
Increment a position in the mu’th direction, looping if needed
\begin{quote}\begin{description}
\item[{Parameters}] \leavevmode\begin{itemize}
\item {} 
\sphinxstyleliteralstrong{\sphinxupquote{ind}} (\sphinxstyleliteralemphasis{\sphinxupquote{int}}) \textendash{} the index of a point on the lattice

\item {} 
\sphinxstyleliteralstrong{\sphinxupquote{mu}} (\sphinxstyleliteralemphasis{\sphinxupquote{int}}) \textendash{} Index corresponding to one of the directions on the lattice:
0:x, 1:y, 2:z, 3:t

\item {} 
\sphinxstyleliteralstrong{\sphinxupquote{La}} (\sphinxstyleliteralemphasis{\sphinxupquote{array\_like}}) \textendash{} Array where each element describes the length of one dimension
of the lattice ({[}x,y,z,t{]})

\end{itemize}

\item[{Returns}] \leavevmode
The function increments a step in the mu’th direction, if the
boundary is met it then loops around to the other side of the
lattice. The return value is the index of the new point on the
lattice.

\item[{Return type}] \leavevmode
numpy.int64

\end{description}\end{quote}

\end{fulllineitems}

\index{p2i() (in module su2)@\spxentry{p2i()}\spxextra{in module su2}}

\begin{fulllineitems}
\phantomsection\label{\detokenize{index:su2.p2i}}\pysiglinewithargsret{\sphinxcode{\sphinxupquote{su2.}}\sphinxbfcode{\sphinxupquote{p2i}}}{\emph{\DUrole{n}{point}}, \emph{\DUrole{n}{La}}}{}
Takes the array describing a point in the spacetime lattice
({[}x,y,z,t{]} notation) and returns the index of that point
\begin{quote}\begin{description}
\item[{Parameters}] \leavevmode\begin{itemize}
\item {} 
\sphinxstyleliteralstrong{\sphinxupquote{point}} (\sphinxstyleliteralemphasis{\sphinxupquote{array\_like}}) \textendash{} Array containing the spacetime coordinates of a position on
the lattice written as {[}x,y,z,t{]}

\item {} 
\sphinxstyleliteralstrong{\sphinxupquote{La}} (\sphinxstyleliteralemphasis{\sphinxupquote{array\_like}}) \textendash{} Array where each element describes the length of one
dimension of the lattice ({[}x,y,z,t{]})

\end{itemize}

\item[{Returns}] \leavevmode
The index of the input point

\item[{Return type}] \leavevmode
int

\end{description}\end{quote}

\end{fulllineitems}

\index{parity() (in module su2)@\spxentry{parity()}\spxextra{in module su2}}

\begin{fulllineitems}
\phantomsection\label{\detokenize{index:su2.parity}}\pysiglinewithargsret{\sphinxcode{\sphinxupquote{su2.}}\sphinxbfcode{\sphinxupquote{parity}}}{\emph{\DUrole{n}{pt}}}{}
Returns the parity of a point on the lattice
\begin{quote}\begin{description}
\item[{Parameters}] \leavevmode
\sphinxstyleliteralstrong{\sphinxupquote{pt}} (\sphinxstyleliteralemphasis{\sphinxupquote{array\_like}}) \textendash{} Point on the lattice, written as {[}x,y,z,t{]}

\item[{Returns}] \leavevmode
Returns 1 if the point has even parity, and 0 if it has odd
parity

\item[{Return type}] \leavevmode
numpy.int64

\end{description}\end{quote}

\end{fulllineitems}

\index{plaq() (in module su2)@\spxentry{plaq()}\spxextra{in module su2}}

\begin{fulllineitems}
\phantomsection\label{\detokenize{index:su2.plaq}}\pysiglinewithargsret{\sphinxcode{\sphinxupquote{su2.}}\sphinxbfcode{\sphinxupquote{plaq}}}{\emph{\DUrole{n}{U}}, \emph{\DUrole{n}{U0i}}, \emph{\DUrole{n}{mups}}, \emph{\DUrole{n}{mu}}, \emph{\DUrole{n}{nu}}}{}
Compute the plaquette
\begin{description}
\item[{U}] \leavevmode{[}array\_like{]}
Array containing the gaugefields for every point on the lattice

\item[{U0i}] \leavevmode{[}int{]}
Lattice point index of the starting point on the lattice for
the calculation.

\item[{mups}] \leavevmode{[}array\_like{]}
The mups array. This array is used as shorthand for taking a
step forwards in the mu’th direction from the U0i’th point

\item[{mu}] \leavevmode{[}int{]}
Index corresponding to one of the directions on the lattice:
0:x, 1:y, 2:z, 3:t

\item[{nu}] \leavevmode{[}int{]}
Index corresponding to another direction on the lattice:
0:x, 1:y, 2:z, 3:t.

\end{description}
\begin{quote}\begin{description}
\item[{Returns}] \leavevmode
The value of the plaquette

\item[{Return type}] \leavevmode
numpy.float64

\end{description}\end{quote}

\end{fulllineitems}

\index{showMc() (in module su2)@\spxentry{showMc()}\spxextra{in module su2}}

\begin{fulllineitems}
\phantomsection\label{\detokenize{index:su2.showMc}}\pysiglinewithargsret{\sphinxcode{\sphinxupquote{su2.}}\sphinxbfcode{\sphinxupquote{showMc}}}{\emph{\DUrole{n}{mat}}, \emph{\DUrole{n}{rc}}}{}
Prints the submatrices along the leftmost column of the full
Dirac matrix
\begin{quote}\begin{description}
\item[{Parameters}] \leavevmode\begin{itemize}
\item {} 
\sphinxstyleliteralstrong{\sphinxupquote{mat}} (\sphinxstyleliteralemphasis{\sphinxupquote{array\_like}}) \textendash{} The Dirac matrix

\item {} 
\sphinxstyleliteralstrong{\sphinxupquote{rc}} (\sphinxstyleliteralemphasis{\sphinxupquote{int}}) \textendash{} The order of the Dirac matrix

\end{itemize}

\item[{Returns}] \leavevmode
Prints 8x8 submatrices that exist along the leftmost section of
the Dirac matrix.

\item[{Return type}] \leavevmode
void

\end{description}\end{quote}

\end{fulllineitems}

\index{showMr() (in module su2)@\spxentry{showMr()}\spxextra{in module su2}}

\begin{fulllineitems}
\phantomsection\label{\detokenize{index:su2.showMr}}\pysiglinewithargsret{\sphinxcode{\sphinxupquote{su2.}}\sphinxbfcode{\sphinxupquote{showMr}}}{\emph{\DUrole{n}{mat}}, \emph{\DUrole{n}{rc}}}{}
Prints the submatrices along the top row of the full Dirac
matrix
\begin{quote}\begin{description}
\item[{Parameters}] \leavevmode\begin{itemize}
\item {} 
\sphinxstyleliteralstrong{\sphinxupquote{mat}} (\sphinxstyleliteralemphasis{\sphinxupquote{array\_like}}) \textendash{} The Dirac matrix

\item {} 
\sphinxstyleliteralstrong{\sphinxupquote{rc}} (\sphinxstyleliteralemphasis{\sphinxupquote{int}}) \textendash{} The order of the Dirac matrix

\end{itemize}

\item[{Returns}] \leavevmode
Prints 8x8 submatrices that exist along the top of the Dirac
matrix.

\item[{Return type}] \leavevmode
void

\end{description}\end{quote}

\end{fulllineitems}

\index{showU() (in module su2)@\spxentry{showU()}\spxextra{in module su2}}

\begin{fulllineitems}
\phantomsection\label{\detokenize{index:su2.showU}}\pysiglinewithargsret{\sphinxcode{\sphinxupquote{su2.}}\sphinxbfcode{\sphinxupquote{showU}}}{\emph{\DUrole{n}{U}}, \emph{\DUrole{n}{mu}}, \emph{\DUrole{n}{i}}}{}
Returns the gauge field at the i’th lattice point and in the
mu’th direction written as matrix writen in the real\sphinxhyphen{}valued
representation of an SU(2) matrix
\begin{quote}\begin{description}
\item[{Returns}] \leavevmode
\begin{itemize}
\item {} 
\sphinxstylestrong{U} (\sphinxstyleemphasis{array\_like}) \textendash{} Real\sphinxhyphen{}valued matrix representaion of an SU(2) gauge field

\item {} 
\sphinxstylestrong{mu} (\sphinxstyleemphasis{int}) \textendash{} Index corresponding to one of the directions on the lattice:
0:x, 1:y, 2:z, 3:t

\item {} 
\sphinxstylestrong{i} (\sphinxstyleemphasis{int}) \textendash{} row index of the Dirac matrix which, when used in conjunction
with j, will locate the first diagonal element of the submatrix

\end{itemize}


\item[{Returns}] \leavevmode
The gauge field which corresponds to the mu’th direction on the
i’th point on the lattice. It is a 2x2 matrix written in the
real\sphinxhyphen{}valued representation

\item[{Return type}] \leavevmode
numpy.ndarray

\end{description}\end{quote}

\end{fulllineitems}

\index{staple() (in module su2)@\spxentry{staple()}\spxextra{in module su2}}

\begin{fulllineitems}
\phantomsection\label{\detokenize{index:su2.staple}}\pysiglinewithargsret{\sphinxcode{\sphinxupquote{su2.}}\sphinxbfcode{\sphinxupquote{staple}}}{\emph{\DUrole{n}{U}}, \emph{\DUrole{n}{U0i}}, \emph{\DUrole{n}{mups}}, \emph{\DUrole{n}{mdns}}, \emph{\DUrole{n}{mu}}, \emph{\DUrole{n}{nu}}, \emph{\DUrole{n}{signnu}}}{}
Compute the staple in the mu\sphinxhyphen{}nu plane
\begin{quote}\begin{description}
\item[{Parameters}] \leavevmode\begin{itemize}
\item {} 
\sphinxstyleliteralstrong{\sphinxupquote{U}} (\sphinxstyleliteralemphasis{\sphinxupquote{array\_like}}) \textendash{} Array containing the gaugefields for every point on the lattice

\item {} 
\sphinxstyleliteralstrong{\sphinxupquote{U0i}} (\sphinxstyleliteralemphasis{\sphinxupquote{int}}) \textendash{} Lattice point index of the starting point on the lattice for
the calculation.

\item {} 
\sphinxstyleliteralstrong{\sphinxupquote{mups}} (\sphinxstyleliteralemphasis{\sphinxupquote{array\_like}}) \textendash{} The mups array. This array is used as shorthand for taking a
step forwards in the mu’th direction from the U0i’th point

\item {} 
\sphinxstyleliteralstrong{\sphinxupquote{mdns}} (\sphinxstyleliteralemphasis{\sphinxupquote{array\_like}}) \textendash{} The mdns array. This array is used as shorthand for taking a
step backwards in the mu’th direction from the U0i’th point

\item {} 
\sphinxstyleliteralstrong{\sphinxupquote{mu}} (\sphinxstyleliteralemphasis{\sphinxupquote{int}}) \textendash{} Index corresponding to one of the directions on the lattice:
0:x, 1:y, 2:z, 3:t

\item {} 
\sphinxstyleliteralstrong{\sphinxupquote{nu}} (\sphinxstyleliteralemphasis{\sphinxupquote{int}}) \textendash{} Index corresponding to one of the directions on the lattice:
0:x, 1:y, 2:z, 3:t. Must be different than mu

\item {} 
\sphinxstyleliteralstrong{\sphinxupquote{signnu}} (\sphinxstyleliteralemphasis{\sphinxupquote{int}}) \textendash{} Equal to either 1 or \sphinxhyphen{}1. Dictates if the staple is calulcuted
forwards (+1) or in reverse (\sphinxhyphen{}1).

\end{itemize}

\item[{Returns}] \leavevmode
Returns the staple starting at the U0i’th point

\item[{Return type}] \leavevmode
numpy.ndarray

\end{description}\end{quote}

\end{fulllineitems}

\index{tr() (in module su2)@\spxentry{tr()}\spxextra{in module su2}}

\begin{fulllineitems}
\phantomsection\label{\detokenize{index:su2.tr}}\pysiglinewithargsret{\sphinxcode{\sphinxupquote{su2.}}\sphinxbfcode{\sphinxupquote{tr}}}{\emph{\DUrole{n}{UU}}}{}
Return the trace of a matrix
\begin{quote}\begin{description}
\item[{Parameters}] \leavevmode
\sphinxstyleliteralstrong{\sphinxupquote{UU}} (\sphinxstyleliteralemphasis{\sphinxupquote{array\_like}}) \textendash{} SU(2) matrix writen in real\sphinxhyphen{}valued form

\item[{Returns}] \leavevmode
The trace of the input matrix

\item[{Return type}] \leavevmode
numpy.float64

\end{description}\end{quote}

\end{fulllineitems}

\index{update() (in module su2)@\spxentry{update()}\spxextra{in module su2}}

\begin{fulllineitems}
\phantomsection\label{\detokenize{index:su2.update}}\pysiglinewithargsret{\sphinxcode{\sphinxupquote{su2.}}\sphinxbfcode{\sphinxupquote{update}}}{\emph{\DUrole{n}{UU}}}{}
Make a random SU(2) matrix near the identity
\begin{quote}\begin{description}
\item[{Parameters}] \leavevmode
\sphinxstyleliteralstrong{\sphinxupquote{UU}} (\sphinxstyleliteralemphasis{\sphinxupquote{array\_like}}) \textendash{} SU(2) matrix written in real\sphinxhyphen{}valued form

\item[{Returns}] \leavevmode
Updated version of the input matrix with a slight random
modification. Matrix is near the identity and written in
real\sphinxhyphen{}valued form

\item[{Return type}] \leavevmode
numpy.ndarray

\end{description}\end{quote}

\end{fulllineitems}

\index{vol() (in module su2)@\spxentry{vol()}\spxextra{in module su2}}

\begin{fulllineitems}
\phantomsection\label{\detokenize{index:su2.vol}}\pysiglinewithargsret{\sphinxcode{\sphinxupquote{su2.}}\sphinxbfcode{\sphinxupquote{vol}}}{\emph{\DUrole{n}{La}}}{}
Takes array of dimensions as input, returns volume
\begin{quote}\begin{description}
\item[{Parameters}] \leavevmode
\sphinxstyleliteralstrong{\sphinxupquote{La}} (\sphinxstyleliteralemphasis{\sphinxupquote{array\_like}}) \textendash{} Array where each element describes the length of one
dimension of the lattice ({[}x,y,z,t{]})

\item[{Returns}] \leavevmode
The volume of the lattice, which is equivalent to the number of
points on the lattice

\item[{Return type}] \leavevmode
int

\end{description}\end{quote}

\end{fulllineitems}

\phantomsection\label{\detokenize{index:module-dirac_generator}}\index{module@\spxentry{module}!dirac\_generator@\spxentry{dirac\_generator}}\index{dirac\_generator@\spxentry{dirac\_generator}!module@\spxentry{module}}\index{gen\_dirac() (in module dirac\_generator)@\spxentry{gen\_dirac()}\spxextra{in module dirac\_generator}}

\begin{fulllineitems}
\phantomsection\label{\detokenize{index:dirac_generator.gen_dirac}}\pysiglinewithargsret{\sphinxcode{\sphinxupquote{dirac\_generator.}}\sphinxbfcode{\sphinxupquote{gen\_dirac}}}{\emph{\DUrole{n}{U}}, \emph{\DUrole{n}{m}}, \emph{\DUrole{n}{spaceLength}}, \emph{\DUrole{n}{timeLength}}, \emph{\DUrole{n}{r}\DUrole{o}{=}\DUrole{default_value}{1.0}}}{}
Generates a set of preconditioned matrices which comprise a
Dirac matrix
\begin{quote}\begin{description}
\item[{Returns}] \leavevmode
\begin{itemize}
\item {} 
\sphinxstylestrong{De} (\sphinxstyleemphasis{array\_like}) \textendash{} A preconditioned matrix containing the elements which connect
an “even” point on the lattice with an “odd” point. Is half the
order of the full Dirac matrix

\item {} 
\sphinxstylestrong{Do} (\sphinxstyleemphasis{array\_like}) \textendash{} A preconditioned matrix containing the elements which connect
an “odd” point on the lattice with an “even” point. All
elements are kinetic terms and De is half the order of the full
Dirac matrix

\item {} 
\sphinxstylestrong{Dm} (\sphinxstyleemphasis{array\_like}) \textendash{} A preconditioned matrix containing the elements containing the
mass terms. Is half the order of the full Dirac matrix.

\end{itemize}


\end{description}\end{quote}

\end{fulllineitems}

\index{invert\_dirac() (in module dirac\_generator)@\spxentry{invert\_dirac()}\spxextra{in module dirac\_generator}}

\begin{fulllineitems}
\phantomsection\label{\detokenize{index:dirac_generator.invert_dirac}}\pysiglinewithargsret{\sphinxcode{\sphinxupquote{dirac\_generator.}}\sphinxbfcode{\sphinxupquote{invert\_dirac}}}{\emph{\DUrole{n}{De}}, \emph{\DUrole{n}{Do}}, \emph{\DUrole{n}{Dm}}, \emph{\DUrole{n}{sizeofinverse}\DUrole{o}{=}\DUrole{default_value}{8}}}{}
Takes in the even/odd preconditioned Dirac matrices and finds
the inverse of the full Dirac matrix
\begin{quote}\begin{description}
\item[{Parameters}] \leavevmode\begin{itemize}
\item {} 
\sphinxstyleliteralstrong{\sphinxupquote{De}} (\sphinxstyleliteralemphasis{\sphinxupquote{array\_like}}) \textendash{} A preconditioned matrix containing the elements which connect
an “even” point on the lattice with an “odd” point. Is half the
order of the full Dirac matrix

\item {} 
\sphinxstyleliteralstrong{\sphinxupquote{Do}} (\sphinxstyleliteralemphasis{\sphinxupquote{array\_like}}) \textendash{} A preconditioned matrix containing the elements which connect
an “odd” point on the lattice with an “even” point. All
elements are kinetic terms and De is half the order of the full
Dirac matrix

\item {} 
\sphinxstyleliteralstrong{\sphinxupquote{Dm}} (\sphinxstyleliteralemphasis{\sphinxupquote{array\_like}}) \textendash{} A preconditioned matrix containing the elements containing the
mass terms. Is half the order of the full Dirac matrix.

\end{itemize}

\item[{Returns}] \leavevmode
The inverse of the Dirac matrix. By default only the first 8
rows of the inverse are calculated.

\item[{Return type}] \leavevmode
numpy.ndarray

\item[{Other Parameters}] \leavevmode
\sphinxstylestrong{sizeofinverse} (\sphinxstyleemphasis{int}) \textendash{} The number of rows of the inverse that will be calculated.

\end{description}\end{quote}

\end{fulllineitems}

\phantomsection\label{\detokenize{index:module-field_gen}}\index{module@\spxentry{module}!field\_gen@\spxentry{field\_gen}}\index{field\_gen@\spxentry{field\_gen}!module@\spxentry{module}}\index{lattice() (in module field\_gen)@\spxentry{lattice()}\spxextra{in module field\_gen}}

\begin{fulllineitems}
\phantomsection\label{\detokenize{index:field_gen.lattice}}\pysiglinewithargsret{\sphinxcode{\sphinxupquote{field\_gen.}}\sphinxbfcode{\sphinxupquote{lattice}}}{\emph{\DUrole{n}{bS}}, \emph{\DUrole{n}{spaceLength}}, \emph{\DUrole{n}{timeLength}}, \emph{\DUrole{n}{M}\DUrole{o}{=}\DUrole{default_value}{1000}}, \emph{\DUrole{n}{Mlink}\DUrole{o}{=}\DUrole{default_value}{10}}, \emph{\DUrole{n}{Msep}\DUrole{o}{=}\DUrole{default_value}{10}}, \emph{\DUrole{n}{Mtherm}\DUrole{o}{=}\DUrole{default_value}{100}}, \emph{\DUrole{n}{bRange}\DUrole{o}{=}\DUrole{default_value}{0.0}}, \emph{\DUrole{n}{bI}\DUrole{o}{=}\DUrole{default_value}{0.1}}, \emph{\DUrole{n}{hotStart}\DUrole{o}{=}\DUrole{default_value}{False}}, \emph{\DUrole{n}{makePlot}\DUrole{o}{=}\DUrole{default_value}{True}}}{}
Generates a lattice or set of lattices utilizing SU(2) gauge fields
\begin{quote}\begin{description}
\item[{Parameters}] \leavevmode\begin{itemize}
\item {} 
\sphinxstyleliteralstrong{\sphinxupquote{bS}} (\sphinxstyleliteralemphasis{\sphinxupquote{float}}) \textendash{} The value of beta that will be used to generate a lattice. If
a set of lattices generated over a range of betas is desired,
then bS is the first value of beta evaluated over the range.

\item {} 
\sphinxstyleliteralstrong{\sphinxupquote{spaceLength}} (\sphinxstyleliteralemphasis{\sphinxupquote{int}}) \textendash{} The length of the three spacial dimensions of the lattice

\item {} 
\sphinxstyleliteralstrong{\sphinxupquote{timeLength}} (\sphinxstyleliteralemphasis{\sphinxupquote{int}}) \textendash{} The length of the time dimension of the lattice

\end{itemize}

\item[{Returns}] \leavevmode
\sphinxstylestrong{th} \textendash{} Numpy array where the fist element is the average value of
the plaquettes of the lattice, and the second element is the
lattice. While the function only returns the final update of
the lattice, the lattice is written out into the configs
directory each time a measurement is made.

\item[{Return type}] \leavevmode
numpy.ndarray

\item[{Other Parameters}] \leavevmode\begin{itemize}
\item {} 
\sphinxstylestrong{M} (\sphinxstyleemphasis{int}) \textendash{} The number of times the lattice is updated. An update of the
lattice is performed by iterating through each lattice point
and each link extending from that point and updating those
links. If enough updates are performed on the lattice it  will
converge to some configuration. Set ot 1000 by default.

\item {} 
\sphinxstylestrong{Mlink} (\sphinxstyleemphasis{int}) \textendash{} The number of times each link is updated per lattice update. A
link is updated by creating a random SU(2) matrix near the
identity, then calculating the action of the new matrix. If the
action is decreased or by some random probability the new
matrix will be assigned to the link, otherwise the link does
not change. This process is repeated Mlink times. Set to 10 by
default

\item {} 
\sphinxstylestrong{Msep} (\sphinxstyleemphasis{int}) \textendash{} The separation of measurements. While the lattice is being
updated M times, a measurement of the average value of the
plaquettes in the lattice will be taken every Msep updates.
Each time a measurement is taken, the lattice will also be
written out to a pickled file.
(E.g. if Msep = 10, measurements will be taken every 10
updates). Set to 10 by default

\item {} 
\sphinxstylestrong{Mtherm} (\sphinxstyleemphasis{int}) \textendash{} The “thermalization” of the lattice. Indicates the number of
lattice updates that need to be performed before any
measurements of the average value of the plaquettes in the
lattice. Set to 100 by default.

\item {} 
\sphinxstylestrong{bRange} (\sphinxstyleemphasis{float}) \textendash{} The difference between the greatest and smallest value of beta
that will be used to generate a lattice. A bRange of 0 means
only one value of beta will be used to make a lattice. Set to
0.0 by default.

\item {} 
\sphinxstylestrong{bI} (\sphinxstyleemphasis{float}) \textendash{} If it is desired to generate a set of lattices using a range of
betas, bI is how much beta will increment between values of
beta. Set to 0.1 by default

\item {} 
\sphinxstylestrong{hotStart} (\sphinxstyleemphasis{bool}) \textendash{} Setting this to True will initialize the fields of the lattice
with a hot start (randomly generated values) rather than the
default cold start (SU(2) matrices set to the identity). Set to
False by default.

\item {} 
\sphinxstylestrong{makePlot} (\sphinxstyleemphasis{bool}) \textendash{} When True, plots of the plaquette vs. updates will be
generated. No plots generated when makePlot is set to False.
Set to True by default.

\end{itemize}

\end{description}\end{quote}

\end{fulllineitems}



\chapter{Indices and tables}
\label{\detokenize{index:indices-and-tables}}\begin{itemize}
\item {} 
\DUrole{xref,std,std-ref}{genindex}

\item {} 
\DUrole{xref,std,std-ref}{modindex}

\item {} 
\DUrole{xref,std,std-ref}{search}

\end{itemize}


\renewcommand{\indexname}{Python Module Index}
\begin{sphinxtheindex}
\let\bigletter\sphinxstyleindexlettergroup
\bigletter{d}
\item\relax\sphinxstyleindexentry{dirac\_generator}\sphinxstyleindexpageref{index:\detokenize{module-dirac_generator}}
\indexspace
\bigletter{f}
\item\relax\sphinxstyleindexentry{field\_gen}\sphinxstyleindexpageref{index:\detokenize{module-field_gen}}
\indexspace
\bigletter{s}
\item\relax\sphinxstyleindexentry{su2}\sphinxstyleindexpageref{index:\detokenize{module-su2}}
\end{sphinxtheindex}

\renewcommand{\indexname}{Index}
\printindex
\end{document}